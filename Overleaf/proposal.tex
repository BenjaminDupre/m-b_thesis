\documentclass{article}
%\usepackage{geometry}
\usepackage{epigraph}
\usepackage[utf8]{inputenc}
%\newgeometry{vmargin={15mm}, hmargin={22mm,27mm}}  

\title{Master Thesis Proposal: Haptic stimuli during a memory-motor an immersive virtual reality (VR) task.}

% \begin{figure}[h!]
% \centering
% \includegraphics[scale=0.35]{bodymind.jpg}
% \caption{What grants that I feel my body as my own?}
% \label{fig:universe}
% \end{figure}

%\epigraph{If you prick us do we not bleed? If you tickle us do we not laugh?}{Shakespeare, Merchant of Venice}
%\date{\parbox{\linewidth}{\centering%
% Mind-Body Relation \endgraf
%  \today\endgraf\newpage}}

\author{Student: Benjamin Dupré\\[1cm]{\small Proposed Advisor: Dr. Gaebler}\\[0cm]{\small Proposed Advisor: Prof. Dr. Villringer}}

\usepackage{natbib}
\usepackage{graphicx}

\begin{document}

\maketitle

%\begin{abstract}
%    Here I, the thesis do. 
%    Thank you
%\end{abstract}

\tableofcontents

\section{Introduction}

By looking into traditionally held concepts in psychology and philosophy of mind, researchers have tapped into newer mental functions that we are mostly unaware of on daily basis. For example, our experience of owning a body and our ability to locate it in space. These somewhat newer functional concepts, are ubiquitous and overlap with traditionally held cognitive functions. 

 The feeling that a part of your body belongs to you, scientist refer to as body ownership. This sensation is a central part of the consciousness experience as it allows us to tell apart the self from the environment\cite{Tsakiris2007}. Experimental work on body ownership shows that body representations are plastic, flexible, and emerge as a result of congruent multisensory integration \cite{kilteni2015}. I will focus on the interoceptive signals signals (i.e., signals that carry physiological conditions of the body \cite{Ceunen2016OnTO}),  as well as proprioceptive signals (i.e., signals that carry the sensation of body position and movement \cite{TUTHILL2018R194}). Here, I intend to address two issues that arise with these signals. (a) Use two alternative measurements of interoceptive signals and (b) observe whether proprioceptive and interoceptive signals work coupled to create the sense of body ownership or not. 
 
  The rubber hand illusion (RHI), embodied illusions have rendered several findings \citep{kilteni2015}. For example, most researchers agree that multisensory integration and synchronicity of stimuli are responsible for eliciting a sense of body ownership \citep{Tsakiris2017}. The synchronicity of exteroceptive signals, such as vision, hearing \& touch, is fundamental for the participants to declare body ownership.  Later on, the illusion extended to a full-body illusion. Here proprioceptive signals, such as mechano-sensitive neurons in muscles, tendons \& joints, became more salient than exteroceptive ones because of the importance in motor correlation \citep{CHANCEL2021104722, Masseli2013}. Likewise, visceral-interoceptive signals are known to play a role in body ownership. Nonetheless, perhaps because a great deal of the interoceptive system functions are unconscious and hard to measure \citep{Brener2016}, how they fit in the greater model is less clear \citep{HORVATH2020361}. In what follows I lay a small review of findings that lead into the hypothesis outline. 
 
 \subsection{ Tactile compared to Haptic}
 USE MY OWN WORDS \newline \\ 
Touch has a very clear precedence over vision and hearing in prenatal
development (see Bremner, Lewkowicz, \& Spence, 2012; Gallace \&
Spence, 2014, for reviews; see Fig. 1A). Tactile sensations are pervasive in how they determine our experiences and behavior in everyday life. Perhaps the first thing which comes to mind when thinking about touch is how we actively bring our skin into contact with objects (usually with our hands or other limbs) in order to encode and recognize objects and their features. This is commonly known as haptics or haptic touch. Tactile receptors can also passively transduce information presented directly to the skin. This function is especially pertinent in the context of conveying the social/interpersonal aspects of touch which in turn bring about strong affective/emotional responses (indeed, these are thought to be particularly important in early life; Field, 2001).


\subsection{ VR compared to Inmersive VR}
PUT IN MY WN WORDS
 Immersion is an objective measure of the vividness offered by a system, and the extent to which the system is capable of shutting out the outside world (Cummings and Bailenson 2016; Slater and Wilbur (1997)). 
 Although the degree of immersion can vary based on the number of senses that are activated by the technology and the quality of the hardware, VR experiences accessed through an HMD or in a CAVE are generally regarded as high immersion. Although the CAMIL is relevant for existing and future immersive learning technologies, and is not a technology-specific theory, in this paper, we focus on immersive learning experiences that are accessed through an HMD (which we refer to as IVR) because most of the recent research has used this technology due to its broad availability. This allows us to provide a concrete description of the process of learning in immersive environments by using a specific technological solution as an example. Simulations or 3D worlds accessed through a desktop computer or tablet are referred to as low immersion or desktop VR in the literature and will only be used as comparisons to IVR in this paper.
(Makransky, G., Petersen, G.B. The Cognitive Affective Model of Immersive Learning (CAMIL): a Theoretical Research-Based Model of Learning in Immersive Virtual Reality. Educ Psychol Rev 33, 937–958 (2021). https://doi.org/10.1007/s10648-020-09586-2)
\subsection{Motor-Memory}


\subsection{VR and interoception/proprioception}


\subsection{Hypothesis}



\section{Materials}

The experimental design methods and materials are all developed in the study NRO-228 Study-DB: 02188.07 - TSVR Akbal/Villringer. All the following material follows from this original study.

\subsection{Experiment:}

After being informed about the experiment the participants did (i) Edinburgh Handedness Questionnaire, (ii) the Virtual Reality Questionnaire and (iii) Cybersickness Questionnaire, (iv) and the Heart Beat Count Task for one minute (HCT). 

After this, the participants were equipped with an HMD and data globes. The (v) VR experiment took place in an immersive virtual environment. In this environment they were sited in front of a virtual table. Here they are given sufficient time to get used to the virtual environment. In front of the virtual room, there was a sketch of the puzzle which they had to memorize and later locate the red ball. During this memorization part, they remained with their virtual hands open, in the palms-up position. The sketch disappeared from the visual field. The red ball was introduced to either one of the hands from the top. They saw a template on the table, which was an identical depiction of the sketch they have seen and memorized at the beginning. The participant had to put the ball in the correct location as fast as possible. In 108 trials, 3 different conditions (relevant haptic stimuli, irrelevant haptic stimuli and base/none haptic stimuli) were equally showed 36 times randomly. The trails appeared rapidly one after the other. The beginning  of the trail was marked by the moment the ball is placed in a hand and the end is marked by the moment the ball reaches the template.

\subsection{VR:} The VR experience displayed and tracked using an HTC Vive HMD, two lighthouses, a leap motion sensor and Haptic Data Globes. Movement is registered from the data-glove, leap motion device and a Head-Set. The data gloves use magnetic tracking sensor for the finger tips although this data will not be consider the movement analysis. It will only consider the wrist which were tracked using the leap-motion device. All movements are registered in a X Y Z euclidean space coordinate system. The original calibrating point (0,0,0) place, giving us 9 streaming sources of data (e.g. Headset X, Headset Y, Headset Z and so on). Rotational data is not considered. 
\subsection{Electrocardiogram:}  The heart rate data is collected using Arduino Uno and a SparkFun Single Lead Heart Rate Monitor - AD8232 integrated into the Unity log file with a 133 HZ frequency.  
\subsection{Questionnaires:} 
\begin{enumerate}
    \item \textbf{Heart beat count task}: The participants silently counted the number of heartbeats for 60 seconds and instructed to report the number at the end of the task. The formula for the score is: \ $ 1 - (Nbeats_{real}-Nbeats_{reported}/((Nbeats_{real}+Nbeats_{reported})/2)$  Yielding one unique score for participant\citep{GARFINKEL201565}.
    \item \textbf{Post-VR Experiment Questionnaire}: is a self-made questionnaire. It consists of 31 items oriented towards capturing whether the experience felt real for the participant or not. It looks into the level of engagement, hand movement, task difficulty and other controlling factors. 
    \item \textbf{Cybersickness Questionnaire}: The version we applied to this study is a shorter adaptation of the simulator sickness questionnaire (SSQ)\citep{SSQ93}. The original version provides straightforward computer or manual scoring, increased power to identify ``problem'' simulators and improved diagnostic capability.
    %\item \textbf{EHQ}: 
\end{enumerate}

\subsection{Analysis:} 
\begin{enumerate}
    \item \textbf{Interoception} I will create interoceptive categories between participants using a median split and group them by high and low interoceptive accuracy roughly following the previous method done by \citep{Tsakiris2011}. For example, indeed, high interoceptive individuals will not change their response time (RT) when irrelevant task haptic stimulation is included OR low interoceptive individuals that will change their RT when irrelevant tactile stimulation is included.
    
    Alternatively, I will group individual interoceptive systems through HRV. This because interoception is thought of as the sensory component of the homeostatic system and HRV indexes, are the primary output of the same homeostatic system in the form of the Autonomous Nervous System  \citep{Pinna2020}. Thus, it is justified to use HRV to generate classes as an alternative measurement of the homeostatic capabilities. To select as a metric the root mean square of successive differences between normal heartbeats (RMSSD) seems the best fit because it consistent within a 5 min time frame of heart data, reflects the beat-to-beat variance in HR and is the primary time-domain measure used to estimate the vagally mediated changes reflected in HRV \cite{Shaffer2017}.
    
    \item \textbf{Movement} Movement styles will be computed using Spatiotemporal Linear Combine distance \citep{Su2020}. It is define as it follows: 

    Given a sample point $p$, a timestamp $p.t$ when $p$ is observed and a trajectory $T$. Where $T = [p_1,p_2,...,p_n]$. The spatial distance is define as $d_{spa}(p,T)= \displaystyle \min_{p'\in T}\{d(p,p')\}$ and temporal distances are defined as $d_{tem}(p,T)= \displaystyle \min_{p'\in T}\{d(|p.t-p'.t|)\}$. To compare two trajctories $T_1$ and $T_2$ their similarities are define as:
    \begin{equation}
        s_{spa}(T_1,T_2)= \frac{\sum_{p\in T_1}e^{-d_{spa}(p,T_2)}}{size(T_1)}+\frac{\sum_{p\in T_2}e^{-d_{spa}(p,T_2)}}{size(T_1)}
    \end{equation}
    \begin{equation}
        s_{tem}(T_1,T_2)= \frac{\sum_{p\in T_1}e^{-d_{tem}(p,T_2)}}{size(T_1)}+\frac{\sum_{p\in T_2}e^{-d_{tem}(p,T_2)}}{size(T_1)}
    \end{equation}
    The spatial and linear similarities are on a range from $[0,2]$ finally the linear combination to compute the spatiotemporal similarities is define as: 
    \begin{equation}
        S_{STLC}(T_1,T_2)= \lambda \cdot s_{spa}(T_1,T_2) + (1-\lambda) \cdot s_{tem}(T_1,T_2)
    \end{equation}
    
    With this, I will create classes of similar movement styles using k-nearest-neighbour (kNN) \citep{Miller2020}. Using these classes we can test the movement predictions. For example, highly efficient movers will not see their RT affected by the inclusion of irrelevant haptic stimuli v/s low-efficiency movers individuals who would have a slower RT. 
\end{enumerate}

\bibliographystyle{plain}
\bibliography{references}
\end{document}
