\documentclass[12pt,oneside,openright]{report}

\usepackage[utf8]{inputenc}
\usepackage[scaled]{helvet}

\renewcommand\familydefault{\sfdefault} 
\usepackage[T1]{fontenc}
\usepackage{fancyhdr,xcolor}

\usepackage{graphicx} % Add the graphicx package for including images
\usepackage{geometry}

\usepackage{xcolor}
\usepackage[style=authoryear]{biblatex} 
\usepackage[colorlinks=true,linkcolor=black,anchorcolor=black,citecolor=black,filecolor=black,menucolor=black,runcolor=black,urlcolor=black]{hyperref}\usepackage{graphicx}
\geometry{
  a4paper,
  left=20mm,
  right=20mm,
  top=4.5cm,
  headheight=4cm,
  bottom=3.5cm,
  footskip=3cm
}

\renewcommand*{\bibfont}{\footnotesize}
\addbibresource{bibliog.bib}
\newcommand{\changefont}{
    \fontsize{18}{16}\selectfont
}
\definecolor{boxcl}{HTML}{1188BB}
\definecolor{tubred}{HTML}{1188BB}



\begin{document}

\begin{titlepage}
    \centering
    % Include the image with a width of one-third of the page
    \includegraphics[width=0.5\textwidth]{Hu-logo.png}
    \vspace{2cm}
    
    {\huge \textbf{Multisensory Integration in Virtual Reality: Effect of Passive Haptic Stimulation}\par}
    \vspace{2cm}
    {\LARGE Master Thesis\par}
    \vspace{0.5cm}
    {\textbf{submitted in fulfillment of the requirements for the degree}\par}
    Master of Science (M.Sc.)\par
    {\textbf{in the master's program ``Mind and Brain''}\par}
    \vspace{1.5cm}
    {\textbf{Humboldt-Universität zu Berlin}\par}
    {\textbf{Berlin School of Mind and Brain}\par}
    \vfill
    \raggedright
    \begin{tabular}{ll}
        \textbf{Handed in by}: & Benjamin Dupré \\
        \textbf{Date of birth:} & 26.04.1986\\
       \textbf{ Address:} & Hoppestraße 16, 13409, Berlin \\
    \end{tabular}
    \vfill
    \begin{tabular}{ll}
        \textbf{1. Supervisor:}& Professor Dr. Arno Villringer \\
        \textbf{2. Supervisor:}& Dr. Michael Gaebler  \\
    \end{tabular}
    \vfill
    {Berlin, den \today \par}
\end{titlepage}


\section*{Introduction}
\subsection*{Problem \& Significance}
explain brain mechanism that predict body changes related to systole

It has been discovered that specific brain mechanisms are responsible for predicting interoceptive bodily changes related to systole, which subsequently dampen the perception of concurrent exteroceptive stimuli within the same time frame \parencite{esra_p,Grund643, motyka}. A parallel phenomenon has been observed in the context of multisensory integration \parencite{SALTAFOSSI2023108642}. This study delves into the influence of the cardiac phase on multisensory integration—a cognitive process facilitating the nonlinear fusion of sensory input to mitigate environmental uncertainty \parencite{SALTAFOSSI2023108642}. Notably, this research not only replicates prior findings but also introduces a multisensory dimension to these mechanisms, thereby making the findings more relevant to everyday life and human perception. However, fully gauging the extent to which this phenomenon holds sway in everyday human psychology poses a complex challenge, as it requires a delicate balance between comprehensiveness and the lucidity of the results.

Furtheremore, the more the modulation between heart cycle and perceptual modulation is establish, the more research ventures into theoretical explanations. One of them being an interoceptive predictive coding process account. For example, using a Markov desicion process (MDP), which is a probailistic generative model, that uses the current cardiac cycle, an visual stimuli it shows that the current observed phenomena could be explained with this computational modell \parencite{Allen2022} Such models is important because they provide the framework to probe or disprobe empirical reasearch that account for psychiatric diseases using Copmutational Phenotyping for example. Nonetheless there are risks to make somewhata quick jump from obervations to a generalized theory that account to psycholgical and human behavioral phenomena. I tool that could allowed for a smooth transition between psichophysics findings and psychological phenomena is Immersive Virtual Reality (IVR)

IVR has proven to be a powerful tool for investigating cognitive processes as it enables researchers to assess behaviors and mental states in complex yet highly controlled scenarios. Traditionally, IVR has relied primarily on visual displays and head-hands movement tracking to create mediated experiences. However, the utilization of VR head-mounted displays in combination with ECG and haptic devices presents new challenges, in practical, technical and in terms of how it agrees with existing literature \cite*{Klotzsche2023}.

Virtual Reality can help us reverse engineer the process. Starting from what we know are wrong assumptions trying to elucidate if the described mechanisms are triggered. To do this we first need to validate the experimental set up by replicating the multisensory findings this far.

Bringing closer experiments closer to ecological validity is relevant because it help us distinguish between what is relevant for the observed behaviour and cognition in the real world.  It often refers to the relation between real-world phenomena and the investigation of these phenomena \parencite{schmuckler2001ecological}. Otherwise we run the risk of not really having clear hor is it that translate from body-brain phenomena into our human psychology.

\subsection*{Thesis Topic \& Goal}

The primary objective of this study is to investigate the feasibility of incorporating touch-cardiac-cycle modulation studies into Interactive Virtual Reality (IVR) setups. IVR, being a system that often involves visual, tactile, and proprioceptive senses, inherently engages multiple senses or is intentionally designed as a multisensory experience. To facilitate a comparative analysis of results, a relevant recent study by Martina Saltafossi on vision, touch, and hearing as multisensory pairs \cite{SALTAFOSSI2023108642} serves as a suitable reference. While there is limited research on two multisensory modalities and none to my knowledge using IVR, this study aims to bridge that gap. However, before delving into specific goals, it is necessary to define the concept of touch, as it encompasses various modes.

The extended classification of tactile sensation \cite{Healy2003HandbookOP} provides a useful framework for understanding touch, categorizing it into five different modes based on the presence or absence of voluntary movement: (1) tactile (cutaneous) perception, (2) passive kinesthetic perception, (3) passive haptic perception, (4) active kinesthetic perception, and (5) active haptic perception. For this thesis, touch is defined as passive haptic perception generated by a vibrating Data-Glove. Based on this definition, three main goals are derived:

\begin{itemize}
  \item[(i)] Assess the impact of passive haptic stimuli on the reported sense of immersion in individuals. This investigation aims to quantify the extent to which passive haptic stimuli influence overall reported scores in questionnaires, shedding light on the role of touch in creating a sense of presence. Saltafossi's study refers to this as "body illusions induced by multisensory conflicts between exteroceptive sensory modalities, such as vision and touch."
    
  \item[(ii)] Evaluate the effect of passive haptic stimuli on performance in the motor-memory task. By examining how passive haptic stimuli influence the response time and accuracy in the motor-memory task, this study seeks to reveal the influence of touch on overall behavioral outcomes in the task.  
    
  \item[(iii)] If the preceding steps yield positive results, we will test if the unlocked-stimuli triggered at diastole or systole has any effect on the response times. This goal involves reproducing existing research that identified modulations in haptic perception synchronized with the cardiac cycle, contributing to the understanding and testing of VR head-mounted displays in combination with ECG and haptic devices.
\end{itemize}

Through an investigation of passive haptic touch's influence on immersion, behavioral outcomes, and interactions with the cardiac cycle, this research aims to enhance our understanding of IVR as a research tool and further validate findings on multisensory integration and perception.


\section*{Materials and Methods}
    \subsection*{Participants}
    \subsection*{Materials}
    \subsection*{Mesurments}
    \subsection*{Statistical Analysis}
\section*{Results}
\section*{Discussion}


%-------- CREATING BIBLIOGRAPHY
\pagebreak

\paragraph{\textbf{References}}
\printbibliography[heading=none]

\end{document}

