\documentclass[12pt,oneside,openright]{report}

\usepackage[utf8]{inputenc}
\usepackage[scaled]{helvet}

\renewcommand\familydefault{\sfdefault} 
\usepackage[T1]{fontenc}
\usepackage{fancyhdr,xcolor}

\usepackage{graphicx} % Add the graphicx package for including images
\usepackage{geometry}

\usepackage{xcolor}
\usepackage[style=authoryear]{biblatex} 
\usepackage[colorlinks=true,linkcolor=black,anchorcolor=black,citecolor=black,filecolor=black,menucolor=black,runcolor=black,urlcolor=black]{hyperref}\usepackage{graphicx}
\geometry{
  a4paper,
  left=20mm,
  right=20mm,
  top=4.5cm,
  headheight=4cm,
  bottom=3.5cm,
  footskip=3cm
}

\renewcommand*{\bibfont}{\footnotesize}
\addbibresource{bibliog.bib}
\newcommand{\changefont}{%
    \fontsize{18}{16}\selectfont
}
\definecolor{boxcl}{HTML}{1188BB}
\definecolor{tubred}{HTML}{1188BB}



\begin{document}

\begin{titlepage}
    \centering
    % Include the image with a width of one-third of the page
    \includegraphics[width=0.5\textwidth]{Hu-logo.png}
    \vspace{2cm}
    
    {\huge \textbf{Multisensory Integration in Virtual Reality: Effect of Passive Haptic Stimulation}\par}
    \vspace{2cm}
    {\LARGE Master Thesis\par}
    \vspace{0.5cm}
    {\textbf{submitted in fulfillment of the requirements for the degree}\par}
    Master of Science (M.Sc.)\par
    {\textbf{in the master's program ``Mind and Brain''}\par}
    \vspace{1.5cm}
    {\textbf{Humboldt-Universität zu Berlin}\par}
    {\textbf{Berlin School of Mind and Brain}\par}
    \vfill
    \raggedright
    \begin{tabular}{ll}
        \textbf{Handed in by}: & Benjamin Dupré \\
        \textbf{Date of birth:} & 26.04.1986\\
       \textbf{ Address:} & Hoppestraße 16, 13409, Berlin \\
    \end{tabular}
    \vfill
    \begin{tabular}{ll}
        \textbf{1. Supervisor:}& Professor Dr. Arno Villringer \\
        \textbf{2. Supervisor:}& Dr. Michael Gaebler  \\
    \end{tabular}
    \vfill
    {Berlin, den \today \par}
\end{titlepage}


\section*{Introduction}

explain brain mechanism that predict body changes related to systole

Its been found that there are brain mechanisms in charge of predicting interoceptive body changes related to systole, that dampen perception of exteroceptive stimuli that happens in the same window of time \parencite{SALTAFOSSI2023108642}. Nonetheless, to correctly asses how significant this phenomena translates in every day is difficult because of the necesity of gradually uncover the phenomema. 

A multisensory study investigated how the cardiac phase modulates multisensory integration, which is the process that allows information from multiple senses to combine non-linearly to reduce environmental uncertainty.On this study previous findings have been replicated and added a multisensory component to this mechanisms, which brought this findings closer to everyday life and perception. 

This observed phenomena has been considered as an example of interoceptive predictive coding process. Explain what is consider and by whoo using quotes HERE. This generalization are problematic since its hard to account for several assumptions or predictions, or were is allocated in the brain. 

Virtual Reality can help us reverse engineer the process. Starting from what we know are wrong assumptions trying to elucidate if the described mechanisms are triggered. To do this we first need to validate the experimental set up by replicating the multisensory findings this far.

Bringing closer experiments closer to ecological validity is relevant because it help us distinguish between what is relevant for the observed behaviour and cognition in the real world and between brain correlations that do not necessarily translate to how our brain w 

%-------- CREATING BIBLIOGRAPHY
\pagebreak

\paragraph{\textbf{References}}
\printbibliography[heading=none]

\end{document}

