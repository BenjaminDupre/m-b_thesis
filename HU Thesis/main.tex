\documentclass[12pt,oneside,openright]{report}

\usepackage[utf8]{inputenc}
\usepackage[scaled]{helvet}

\renewcommand\familydefault{\sfdefault} 
\usepackage[T1]{fontenc}
\usepackage{fancyhdr,xcolor}

\usepackage{graphicx} % Add the graphicx package for including images
\usepackage{geometry}

\usepackage{xcolor}
\usepackage[style=authoryear,backend=biber]{biblatex}
\addbibresource{bibliog.bib}
\usepackage[colorlinks=true,linkcolor=black,anchorcolor=black,citecolor=black,filecolor=black,menucolor=black,runcolor=black,urlcolor=black]{hyperref}\usepackage{graphicx}
\geometry{
  a4paper,
  left=20mm,
  right=20mm,
  top=4.5cm,
  headheight=4cm,
  bottom=3.5cm,
  footskip=3cm
}

\renewcommand*{\bibfont}{\footnotesize}

\newcommand{\changefont}{
    \fontsize{18}{16}\selectfont
}
\definecolor{boxcl}{HTML}{1188BB}
\definecolor{tubred}{HTML}{1188BB}



\begin{document}

\begin{titlepage}
    \centering
    % Include the image with a width of one-third of the page
    \includegraphics[width=0.5\textwidth]{Hu-logo.png}
    \vspace{2cm}
    
    {\huge \textbf{Multisensory Integration in Virtual Reality: Effect of Passive Haptic Stimulation}\par}
    \vspace{2cm}
    {\LARGE Master Thesis\par}
    \vspace{0.5cm}
    {\textbf{submitted in fulfillment of the requirements for the degree}\par}
    Master of Science (M.Sc.)\par
    {\textbf{in the master's program ``Mind and Brain''}\par}
    \vspace{1.5cm}
    {\textbf{Humboldt-Universität zu Berlin}\par}
    {\textbf{Berlin School of Mind and Brain}\par}
    \vfill
    \raggedright
    \begin{tabular}{ll}
        \textbf{Handed in by}: & Benjamin Dupré \\
        \textbf{Date of birth:} & 26.04.1986\\
       \textbf{ Address:} & Hoppestraße 16, 13409, Berlin \\
    \end{tabular}
    \vfill
    \begin{tabular}{ll}
        \textbf{1. Supervisor:}& Professor Dr. Arno Villringer \\
        \textbf{2. Supervisor:}& Dr. Michael Gaebler  \\
    \end{tabular}
    \vfill
    {Berlin, den \today \par}
\end{titlepage}

\section*{Introduction}
\subsection*{Problem \& Significance}

After a long run, you may have experienced time seemingly slowing down immediately after stopping \parencite{Edwards2017TimePP}. Could heart rhythm be a factor in this altered perception? 

Research in interoception has shown that visceral signals can influence how we process exteroceptive information. Specific brain mechanisms responsible for predicting signals from within the body, including interoceptive body changes related to heart systole, play a role in diminishing the perception of external body signals, such as touch, vision, and auditory stimuli \parencite{esra_p, AL2021118247, Grund643, motyka, Park2014}. This phenomenon has been well-established in the context individual sense modalities and only recently in a multisensory experiment.

The multisensory integration experiment conducted by Saltafossi found that \parencite{SALTAFOSSI2023108642}. This study examines the influence of the cardiac phase on multisensory integration, a cognitive process that facilitates the nonlinear fusion of sensory input to reduce environmental uncertainty.

However task relevant information is not presented in a phased-locked manner or passively selected. It has also been stablsihed that more ee movement is generated during the systolic phase of the cardiac cycle and more fixation during the diatole phase \parencite{GalvezPol2018ActiveSI}

Notably, this research replicates the previously mentioned findings and introduces a multisensory dimension to these mechanisms. This makes the findings more relevant to everyday life and human perception. However, accurately assessing the extent to which this phenomenon influences everyday experiences, such as how we feel after running, is challenging. It requires a delicate balance between obtaining clear psychophysical results and integrating them with other cognitive functions.

Furthermore, as the connection between the heart cycle and perceptual modulation becomes more established, research delves into theoretical explanations. One such explanation is an interoceptive predictive coding process. For example, using a Markov decision process (MDP), which is a probabilistic generative model that utilizes the current cardiac cycle and visual stimuli, it shows that the observed phenomena could be explained with this computational model \parencite{Allen2022}. Such models are important because they provide a framework to test or disprove empirical research that accounts for psychiatric diseases using computational phenotyping, for example. Nevertheless, there are risks associated with making a hasty leap from observations to a generalized theory that accounts for psychological and human behavioral phenomena. A tool that can facilitate a smooth transition between psychophysics findings and psychological phenomena is Immersive Virtual Reality (IVR).

IVR has proven to be a powerful tool for investigating cognitive processes as it enables researchers to assess behaviors and mental states in complex yet highly controlled scenarios. Traditionally, IVR has relied primarily on visual displays and head-hand movement tracking to create mediated experiences. However, the utilization of VR head-mounted displays in combination with ECG and haptic devices presents new challenges, both in practical and technical terms, and in terms of how it aligns with existing literature \parencite{Klotzsche2023}.

Virtual Reality can help us reverse-engineer the process, starting from what we know are incorrect assumptions and attempting to elucidate if the described mechanisms are triggered. To do this, we first need to validate the experimental setup by replicating the multisensory findings obtained thus far.

Bringing experiments closer to ecological validity is relevant because it helps us distinguish what is relevant for observed behavior and cognition in the real world. It often refers to the relationship between real-world phenomena and the investigation of these phenomena \parencite{schmuckler2001ecological}. Otherwise, we run the risk of not truly understanding how body-brain phenomena translate into our human psychology.



\subsection*{Thesis Topic \& Goal}

The primary objective of this study is to investigate the feasibility of incorporating touch-cardiac-cycle modulation studies into Interactive Virtual Reality (IVR) setups. IVR, being a system that often involves visual, tactile, and proprioceptive senses, inherently engages multiple senses or is intentionally designed as a multisensory experience. To facilitate a comparative analysis of results, a relevant recent study by Martina Saltafossi on vision, touch, and hearing as multisensory pairs \cite{SALTAFOSSI2023108642} serves as a suitable reference. While there is limited research on two multisensory modalities and none to my knowledge using IVR, this study aims to bridge that gap. However, before delving into specific goals, it is necessary to define the concept of touch, as it encompasses various modes.

The extended classification of tactile sensation \cite{Healy2003HandbookOP} provides a useful framework for understanding touch, categorizing it into five different modes based on the presence or absence of voluntary movement: (1) tactile (cutaneous) perception, (2) passive kinesthetic perception, (3) passive haptic perception, (4) active kinesthetic perception, and (5) active haptic perception. For this thesis, touch is defined as passive haptic perception generated by a vibrating Data-Glove. Based on this definition, three main goals are derived:

\begin{itemize}
  \item[(i)] Assess the impact of passive haptic stimuli on the reported sense of immersion in individuals. This investigation aims to quantify the extent to which passive haptic stimuli influence overall reported scores in questionnaires, shedding light on the role of touch in creating a sense of presence. Saltafossi's study refers to this as "body illusions induced by multisensory conflicts between exteroceptive sensory modalities, such as vision and touch."
    
  \item[(ii)] Evaluate the effect of passive haptic stimuli on performance in the motor-memory task. By examining how passive haptic stimuli influence the response time and accuracy in the motor-memory task, this study seeks to reveal the influence of touch on overall behavioral outcomes in the task.  
    
  \item[(iii)] If the preceding steps yield positive results, we will test if the unlocked-stimuli triggered at diastole or systole has any effect on the response times. This goal involves reproducing existing research that identified modulations in haptic perception synchronized with the cardiac cycle, contributing to the understanding and testing of VR head-mounted displays in combination with ECG and haptic devices.
\end{itemize}

Through an investigation of passive haptic touch's influence on immersion, behavioral outcomes, and interactions with the cardiac cycle, this research aims to enhance our understanding of IVR as a research tool and further validate findings on multisensory integration and perception.


\section*{Materials and Methods}
    \subsection*{Participants}
    \subsection*{Materials}
    \subsection*{Mesurments}
    \subsection*{Statistical Analysis}
\section*{Results}
\section*{Discussion}


%-------- CREATING BIBLIOGRAPHY
\pagebreak

%\bibliographystyle{plain} % Choose your style
%\bibliography{bibliog.bib}  % Use your actual .bib file name

\paragraph{\textbf{References}}
\printbibliography[heading=none]

\end{document}

