\documentclass{article}
%\usepackage{geometry}
\usepackage{epigraph}
\usepackage[utf8]{inputenc}
%\newgeometry{vmargin={15mm}, hmargin={22mm,27mm}}  

%\title{ Passive haptic stimuli during a memory-motor task in immersive virtual reality.}
\title{Request for a non Mind\&Brain supervisor.}
\vspace{2cm}
\author{
    {Proposed Advisor: Dr. Michael Gaebler}\\[0cm]
    {Max Planck Institute for Human Cognitive and Brain Sciences}\\[0,4cm]
    {\small Student: Benjamin Dupré}\\[0cm]
    {\small Mind and Brain}} %\\[0cm]{\small Proposed Advisor: Prof. Dr. Villringer}}

% \begin{figure}[h!]
% \centering
% \includegraphics[scale=0.35]{bodymind.jpg}
% \caption{What grants that I feel my body as my own?}
% \label{fig:universe}
% \end{figure}

%\epigraph{If you prick us do we not bleed? If you tickle us do we not laugh?}{Shakespeare, Merchant of Venice}
%\date{\parbox{\linewidth}{\centering%
% Mind-Body Relation \endgraf
%  \today\endgraf\newpage}}



\usepackage{natbib}
\usepackage{graphicx}

\begin{document}



\maketitle
\pagebreak
%\begin{abstract}
%    Here I, the thesis do. 
%    Thank you
%\end{abstract}

%\tableofcontents
 
\section{Introduction}

In psychology, the sense of Touch has been overshadow by the sense Sight---and that this is hindering because in fact Touch is central to the development of multisensory expirience\cite{BREMNER2017227}.

Inmersive Virtual Reality (IVR) offers the rear oportunity to understand better how is it that Touch is central to the multyesensory expirience in valid and ecological environment. IN this study also  technology offers the posibility of exploiting the fact that at the moment inmersive virtual reality can mediate expirience with higher visual resolution than the lower haptic gloves. 

This thesis wants to determine how much does Haptic stimuli affect the performance in a motor-memory task and inmersivness expirience reported by the participants. By doing this contribute to our understanding of the Sense of Touch in the overall feeling of embodiedment. 
 
 \subsection{ Tactile compared to Haptic}
 USE MY OWN WORDS \newline \\

Within the research of Touch its considered by some authors that there is an importante distintion between Tactile and Haptic.

When we include Touch into the mediated experiment we need to differentiate between Touch and Haptic Stimuli. 



make the difference between touch and haptic stimuli because as a matter of   
Touch has a very clear precedence over vision and hearing in prenatal
development \citep{bremner2012multisensory}. Tactile sensations are pervasive in how they determine our experiences and behavior in everyday life. Perhaps the first thing which comes to mind when thinking about touch is how we actively bring our skin into contact with objects (usually with our hands or other limbs) in order to encode and recognize objects and their features. This is commonly known as haptics or haptic touch. Tactile receptors can also passively transduce information presented directly to the skin. This function is especially pertinent in the context of conveying the social/interpersonal aspects of touch which in turn bring about strong affective/emotional responses (indeed, these are thought to be particularly important in early life; Field, 2001).


\subsection{ VR compared to Inmersive VR}
PUT IN MY WN WORDS \newline \\ 
 Immersion is an objective measure of the vividness offered by a system, and the extent to which the system is capable of shutting out the outside world (Cummings and Bailenson 2016; Slater and Wilbur (1997)). 
 Although the degree of immersion can vary based on the number of senses that are activated by the technology and the quality of the hardware, VR experiences accessed through an HMD or in a CAVE are generally regarded as high immersion. Although the CAMIL is relevant for existing and future immersive learning technologies, and is not a technology-specific theory, in this paper, we focus on immersive learning experiences that are accessed through an HMD (which we refer to as IVR) because most of the recent research has used this technology due to its broad availability. This allows us to provide a concrete description of the process of learning in immersive environments by using a specific technological solution as an example. Simulations or 3D worlds accessed through a desktop computer or tablet are referred to as low immersion or desktop VR in the literature and will only be used as comparisons to IVR in this paper.
(Makransky, G., Petersen, G.B. The Cognitive Affective Model of Immersive Learning (CAMIL): a Theoretical Research-Based Model of Learning in Immersive Virtual Reality. Educ Psychol Rev 33, 937–958 (2021). https://doi.org/10.1007/s10648-020-09586-2)
\subsection{Motor-Memory}


\subsection{VR and interoception/proprioception}


\subsection{Hypothesis}



\section{Materials}

The experimental design methods and materials are all developed in the study NRO-228 Study-DB: 02188.07 - TSVR Akbal/Villringer. All the following material follows from this original study.

\subsection{Experiment:}

After being informed about the experiment the participants did (i) Edinburgh Handedness Questionnaire, (ii) the Virtual Reality Questionnaire and (iii) Cybersickness Questionnaire, (iv) and the Heart Beat Count Task for one minute (HCT). 

After this, the participants were equipped with an HMD and data globes. The (v) VR experiment took place in an immersive virtual environment. In this environment they were sited in front of a virtual table. Here they are given sufficient time to get used to the virtual environment. In front of the virtual room, there was a sketch of the puzzle which they had to memorize and later locate the red ball. During this memorization part, they remained with their virtual hands open, in the palms-up position. The sketch disappeared from the visual field. The red ball was introduced to either one of the hands from the top. They saw a template on the table, which was an identical depiction of the sketch they have seen and memorized at the beginning. The participant had to put the ball in the correct location as fast as possible. In 108 trials, 3 different conditions (relevant haptic stimuli, irrelevant haptic stimuli and base/none haptic stimuli) were equally showed 36 times randomly. The trails appeared rapidly one after the other. The beginning  of the trail was marked by the moment the ball is placed in a hand and the end is marked by the moment the ball reaches the template.

\subsection{VR:} The VR experience displayed and tracked using an HTC Vive HMD, two lighthouses, a leap motion sensor and Haptic Data Globes. Movement is registered from the data-glove, leap motion device and a Head-Set. The data gloves use magnetic tracking sensor for the finger tips although this data will not be consider the movement analysis. It will only consider the wrist which were tracked using the leap-motion device. All movements are registered in a X Y Z euclidean space coordinate system. The original calibrating point (0,0,0) place, giving us 9 streaming sources of data (e.g. Headset X, Headset Y, Headset Z and so on). Rotational data is not considered. 
\subsection{Electrocardiogram:}  The heart rate data is collected using Arduino Uno and a SparkFun Single Lead Heart Rate Monitor - AD8232 integrated into the Unity log file with a 133 HZ frequency.  
\subsection{Questionnaires:} 
\begin{enumerate}
    \item \textbf{Heart beat count task}: The participants silently counted the number of heartbeats for 60 seconds and instructed to report the number at the end of the task. The formula for the score is: \ $ 1 - (Nbeats_{real}-Nbeats_{reported}/((Nbeats_{real}+Nbeats_{reported})/2)$  Yielding one unique score for participant\citep{GARFINKEL201565}.
    \item \textbf{Post-VR Experiment Questionnaire}: is a self-made questionnaire. It consists of 31 items oriented towards capturing whether the experience felt real for the participant or not. It looks into the level of engagement, hand movement, task difficulty and other controlling factors. 
    \item \textbf{Cybersickness Questionnaire}: The version we applied to this study is a shorter adaptation of the simulator sickness questionnaire (SSQ)\citep{SSQ93}. The original version provides straightforward computer or manual scoring, increased power to identify ``problem'' simulators and improved diagnostic capability.
    %\item \textbf{EHQ}: 
\end{enumerate}

\subsection{Analysis:} 
\begin{enumerate}
    \item \textbf{Interoception} I will create interoceptive categories between participants using a median split and group them by high and low interoceptive accuracy roughly following the previous method done by \citep{Tsakiris2011}. For example, indeed, high interoceptive individuals will not change their response time (RT) when irrelevant task haptic stimulation is included OR low interoceptive individuals that will change their RT when irrelevant tactile stimulation is included.
    
    Alternatively, I will group individual interoceptive systems through HRV. This because interoception is thought of as the sensory component of the homeostatic system and HRV indexes, are the primary output of the same homeostatic system in the form of the Autonomous Nervous System  \citep{Pinna2020}. Thus, it is justified to use HRV to generate classes as an alternative measurement of the homeostatic capabilities. To select as a metric the root mean square of successive differences between normal heartbeats (RMSSD) seems the best fit because it consistent within a 5 min time frame of heart data, reflects the beat-to-beat variance in HR and is the primary time-domain measure used to estimate the vagally mediated changes reflected in HRV \cite{Shaffer2017}.    
\end{enumerate}

\bibliographystyle{plain}
\bibliography{references}
\end{document}
