\documentclass[12pt,oneside,openright]{report}
%Document Variables
\newcommand{\topic}{PPPPPPP}
\newcommand{\kind}{Master}
\newcommand{\supervisor}{Michael,Gaebler --- gaebler@cbs.mpg.de --- \href{https://www.cbs.mpg.de/person/gaebler/371395}{Web Page}.

}
\newcommand{\skills}{Passive Haptic Feedback, Inmersive Virtual Reality, Touch Perception}

\usepackage[utf8]{inputenc}
\usepackage[scaled]{helvet}
\renewcommand\familydefault{\sfdefault} 
\usepackage[T1]{fontenc}
\usepackage{fancyhdr,xcolor}

\usepackage{xcolor}
\usepackage{biblatex}
\usepackage[colorlinks=true,linkcolor=black,anchorcolor=black,citecolor=black,filecolor=black,menucolor=black,runcolor=black,urlcolor=black]{hyperref}\usepackage{graphicx}
\usepackage{geometry}
\geometry{
  a4paper,
  left=20mm,
  right=20mm,
  top=4.5cm,
  headheight=4cm,
  bottom=3.5cm,
  footskip=3cm
}

\renewcommand*{\bibfont}{\footnotesize}
\addbibresource{references.bib}
\newcommand{\changefont}{%
    \fontsize{18}{16}\selectfont
}
\definecolor{boxcl}{HTML}{1188BB}
\definecolor{tubred}{HTML}{1188BB}


\let\oldheadrule\headrule% Copy \headrule into \oldheadrule
\renewcommand{\headrule}{\color{tubred}\oldheadrule}% Add colour to \headrule
\renewcommand{\headrulewidth}{1.5pt}
\fancyfoot{}
\fancyhead[HL]{\parbox{0.80\textwidth}{{\changefont \textbf{\kind{}’s Thesis} \topic{}}}}
\fancyhead[HR]{\includegraphics[width=0.38\textwidth]{hu_siegel-kombi_rgb.png}}
%\fancyfoot[R]{\colorbox{boxcl}{\parbox[b][][r]{0.55\textwidth}{\textcolor{white}{\textbf{Contact:} \supervisor}}}}
\pagestyle{fancy}
\begin{document}

\section*{Summary:}
Summary here

\vspace*{0,5cm}

\paragraph{\textbf{Keywords:}}\skills{}


\section*{1. Problem \& Significance}

Behavioral and mechanistic studies of perception often prioritize the examination of the brain \cite{Hofmann2021}, neglecting the impact of bodily signals on perception and cognition. To achieve a comprehensive understanding of how our bodies shape our cognitive processes, it is imperative to delve deeper into the sense of touch. Touch plays a vital role to undertand embodied perception as it gathers information from the entire body \cite*{Field2014}.  Furthermore, touch interacts with coupled respiratory-cardiac cycles, optimizing tactile performance \cite{Grund643}.  Additionally, touch significantly contributes to the development of multisensory experiences \cite{BREMNER2017227} and notably, touch is instrumental in our brain's ability to sustain neural representations of the body through tactile feedback \cite{Cole2016}.

In our pursuit of fully comprehending perceptual phenomena, we have turned to Immersive Virtual Reality (IVR). IVR has proven to be a powerful tool for investigating cognitive processes as it enables researchers to assess behaviors and mental states in complex yet highly controlled scenarios. Traditionally, IVR has relied primarily on visual displays and head-hands movement tracking to create mediated experiences. However, the utilization of VR head-mounted displays in combination with ECG and haptic devices presents new challenges, both practical and in terms of generalizability of previous findings to a VR-haptic setup \cite*{Klotzsche2023}.

Can mediated experiences of touch evoke similar results to those observed in embodied cognition? By understanding which aspects of touch can be mediated and how they interact with other mediated senses, we can confidently advance experimental setups to explore the body-mind relationship with greater ecological validity. Realizing which components of touch can be altered and mediated in IVR will allow researchers to further validate current findings in the field of embodied cognition and will give to the industry gide-lines as to which hardwear-devices develope further.  

\section*{2. Thesis Topic \& Goal}

This study aims to explore if replicating studies on haptic-cardiac-cycle modulation is possible in IVR setups. For this I use as reference the extended classification of tactile sensation, which categorizes touch into five different modes based on the presence or absence of voluntary movement: (1) tactile (cutaneous) perception, (2) passive kinesthetic perception, (3) passive haptic perception, (4) active kinesthetic perception, and (5) active haptic perception \cite{Healy2003HandbookOP}. This Thesis focuses on mode of Touch three. 

The primary goals of the thesis are as follows:
  \begin{itemize}
    \item[(i)] Determine the \textbf{impact of passive haptic stimuli on the reported sense of immersion experienced by individuals}. This investigation aims to quantify the extent to which passive haptic stimuli influence the overall reported score in questionnaire, providing insights into the role of touch in creating a sense of presence.
    \item[(ii)] Assess the \textbf{impact of passive haptic stimuli on the behavioral outcomes in IVR}. By examining how passive haptic stimuli influence (a) repose time and (b) accuracy in the motor-memory task. Additionally, we will observe if there is a (c) difference in movement between each condition. This goal aims to uncover influence in behavioral outcomes, shedding light on the potential benefits of incorporating haptic feedback in motor-task.
    \item[(iii)] Only if the are significant differences in behavioral outcomes in IVR, \textbf{measure if passive haptic feedback initiated in specific cardiac-cycles correlate with differences on the behavioral outcomes in IVR}. This would validate the possibility of designing experiments in IVR that delve into haptic perception synchronized with the cardiac cycle. Thus, Contributing to understand and test VR head-mounted displays in combination with ECG and haptic devices.
  \end{itemize}

  The thesis aims to enhance our understanding of the sense of touch in relation to embodied sensations and the overall immersive quality of mediated touch. By exploring the influence of passive haptic touch against immersion, behavioral outcomes, and interactions with the cardiac cycle, this research intends to contribute to a broader understanding of the intricate relationship between touch, embodiment, and immersive experiences.

\subsection*{3 Methods}
The experimental design methods and materials employed in this study were developed based on the NRO-228 Study-DB: 02188.07 - TSVR Akbal/Villringer. While I did not contribute to the experimental design phase, my involvement encompassed the study implementation, data collection and anylysis. Nevertheless, it is important to note that not all the data collected during the original study will be utilized in this master's thesis. Specifically, the heartbeat count task will be excluded to ensure a concentrated focus on the specific objectives of this thesis. Furthermore, the hypothesis and analysis presented in this thesis are distinct and tailored to this research project.

\subsubsection*{3.1 Participants}
CHECK NUMBERS OF PARTICIPANTS There were a total of 22 participans who performed both of the full tasks. Nonetheless, 1 of them did not filled all questionaires therefore we had to leave them out and continue to performe the analysis with 21 participants. The call for participants was answer considerably more by women (17) than men (5). The age accorss the sample was consistenly around 25 years old, all but for one exception ($\mu=25.1 , \sigma=6.3$).

\subsubsection*{3.2 Materials}
  \begin{enumerate}
    \item[3.2.1] \textbf{Electrocardiogram:} Heart rate data is collected using an Arduino Uno and a SparkFun Single Lead Heart Rate Monitor - AD8232. The collected data is collected through a USB 2.0 and integrated into the Unity log file at a frequency of 133 Hz.
    \item[3.2.2] \textbf{Head Mounted Display \& Lighthouse:} HTC Vive head-mounted display (HMD) with  two lighthouses.The headset specificiations include a Dual AMOLED 3.6" diagonal, 	1080 x 1200 pixels per eye (2160 x 1200 pixels combined), 90 Hz refresh and a 110 degrees field of view.  The lighthouse are 2 SteamVR Tracking, G-sensor, gyroscope, proximity. Both connected using  USB 2.0. We did not use the controllers, the hands were tracked using the leap motion sensor. 
    \item[3.2.3] \textbf{Leap Motion Controller:}  The field of view is 150x120 degrees with a variable range of roughly 80 cm (arms length), with a weight of 32 grams placed on the HMD. The device has two 640x240 infrared cameras with a frame rate of 120 fps. 
    \item[3.2.3] \textbf{Data-gloves:}  These gloves feature magnetic sensors and are connected to Unity using a microUSB connection. The gloves offer haptic feedback through 10 vibrotactile actuators, which provide a wide range of tactile sensations with 1,024 levels of intensity. They also incorporate complete finger tracking with six 9-axis Inertial Measurement Units (IMUs). These IMUs enable precise tracking of finger movements, allowing for accurate gesture recognition and enhanced interaction in virtual environments.
  \end{enumerate}

\subsubsection*{3.3 Measures}
The materials were the following:
  \begin{enumerate}
    \item[3.3.1] \textbf{Experiment:} Upon receiving information about the experiment, participants completed the following tasks:
    \begin{enumerate}
      \item[(i)] Edinburgh Handedness Questionnaire,
      \item[(ii)] Virtual Reality Questionnaire,
      \item[(iii)] Cybersickness Questionnaire,
      \item[(iv)] One minute Heart Beat Count Task (HCT)  and
      \item[(v)] Immersive Virtual Reality Task   
    \end{enumerate}
    
    \item[3.3.2] \textbf{Imersive Virtual Reality:} The VR experience was presented and tracked using an HTC Vive head-mounted display (HMD), two lighthouses, a Leap Motion sensor, and Haptic Data Globes. Movement data was collected from the data gloves, Leap Motion device, and the HMD. The data gloves utilized a magnetic tracking sensor for the fingertips, although this data was not considered for the movement analysis. Only the wrist movements, tracked by the Leap Motion device, were taken into account. All movements were recorded in a Euclidean coordinate system (X, Y, Z) \footnote{Future research could focus on the change of behaviour on the index finger \cite{Shao2016}}. The original calibrating point was set at (0, 0, 0), providing a total of nine streaming sources of data (e.g., Headset X, Headset Y, Headset Z, and so on). Rotational data was not included in the analysis.



    \item[3.3.4] \textbf{Questionnaires:}
      \begin{enumerate}
        \item[(i)] Post-VR Experiment Questionnaire: is a self-made questionnaire. It consists of 31 items oriented towards capturing whether the experience felt real for the participant or not. It looks into the level of engagement, hand movement, task difficulty and other controlling factors. 
        \item[(ii)] Cybersickness Questionnaire: The version we applied to this study is a shorter adaptation of the simulator sickness questionnaire (SSQ) \cite*{SSQ93}. The original version provides straightforward computer or manual scoring, increased power to identify problem simulators and improved diagnostic capability.
    \end{enumerate}
  \end{enumerate}

\subsubsection*{3.3 Procedure}
After completing these tasks, participants were equipped with a head-mounted display (HMD) and data gloves. The VR experiment took place within an immersive virtual environment. Within this environment, participants were positioned in front of a virtual table and given sufficient time to acclimate to the virtual surroundings. A sketch of a puzzle, which they needed to memorize, was displayed in front of the virtual room. During the memorization phase, participants kept their virtual hands open with palms facing up. Subsequently, the sketch disappeared from their visual field.

 A red ball was then introduced from the top, appearing in either one of the hands. Participants observed a template on the table, which was an identical representation of the initial sketch they had memorized. Their task was to place the ball in the correct location on the template as quickly as possible. Throughout 108 trials, three different conditions (relevant haptic stimuli, irrelevant haptic stimuli, and no haptic stimuli) were randomly presented an equal number of times (36 times each). The trials appeared rapidly, one after the other. The start of each trial was marked by placing the ball in a hand, and the end of the trial was marked by the moment the ball reached the designated location on the template.


\vspace*{3cm}
\hfill  
\begin{center}
  Benjamin Dupré \\ 
  \includegraphics[width=4cm]{firma.png}
\end{center}

\pagebreak
\paragraph{\textbf{References}}
\printbibliography[heading=none]



\end{document}
