\documentclass[12pt,oneside,openright]{report}
%Document Variables
\newcommand{\topic}{Passive haptic stimuli \& cardiac cycle modulation in a virtual reality setup.}
\newcommand{\kind}{Master Thesis Proposal: }
\newcommand{\supervisor}{Michael,Gaebler --- gaebler@cbs.mpg.de --- \href{https://www.cbs.mpg.de/person/gaebler/371395}{Web Page}.

}
\newcommand{\skills}{Passive Haptic Feedback, Inmersive Virtual Reality, Touch Perception}

\usepackage[utf8]{inputenc}
\usepackage[scaled]{helvet}

\renewcommand\familydefault{\sfdefault} 
\usepackage[T1]{fontenc}
\usepackage{fancyhdr,xcolor}

\usepackage{xcolor}
\usepackage{biblatex}
\usepackage[colorlinks=true,linkcolor=black,anchorcolor=black,citecolor=black,filecolor=black,menucolor=black,runcolor=black,urlcolor=black]{hyperref}\usepackage{graphicx}
\usepackage{geometry}
\geometry{
  a4paper,
  left=20mm,
  right=20mm,
  top=4.5cm,
  headheight=4cm,
  bottom=3.5cm,
  footskip=3cm
}

%\usepackage{times}

\renewcommand*{\bibfont}{\footnotesize}
\addbibresource{references.bib}
\newcommand{\changefont}{%
    \fontsize{18}{16}\selectfont
}
\definecolor{boxcl}{HTML}{1188BB}
\definecolor{tubred}{HTML}{1188BB}


\let\oldheadrule\headrule% Copy \headrule into \oldheadrule
\renewcommand{\headrule}{\color{tubred}\oldheadrule}% Add colour to \headrule
\renewcommand{\headrulewidth}{1.5pt}
\fancyfoot{}
\fancyhead[HL]{\parbox{0.80\textwidth}{{\changefont \textbf{\kind{}} \topic{}}}}
\fancyhead[HR]{\includegraphics[width=0.38\textwidth]{hu_siegel-kombi_rgb.png}}
%\fancyfoot[R]{\colorbox{boxcl}{\parbox[b][][r]{0.55\textwidth}{\textcolor{white}{\textbf{Contact:} \supervisor}}}}
\pagestyle{fancy}
\begin{document}

\section*{1. Problem \& Significance}

%Behavioral and mechanistic studies of perception often prioritize the examination of the brain, neglecting the impact of bodily signals on perception and cognition . 
To achieve a comprehensive understanding of how our bodies shape our cognitive processes, it is imperative to delve deeper into the sense of touch\cite{Hofmann2021}. 

Touch plays a vital role in embodied perception as it gathers information from the entire body \cite*{Field2014}.  Furthermore, touch perception interacts with coupled respiratory-cardiac cycles, optimizing tactile performance \cite{Grund643,esra,motyka}.  Additionally, touch significantly contributes to the development of multisensory experiences \cite{BREMNER2017227,SALTAFOSSI2023108642} and notably, touch is instrumental in our brain's ability to sustain neural representations of the body through tactile feedback \cite{Cole2016}.

In our pursuit of fully comprehending perceptual phenomena, we have turned to Immersive Virtual Reality (IVR). IVR has proven to be a powerful tool for investigating cognitive processes as it enables researchers to assess behaviors and mental states in complex yet highly controlled scenarios. Traditionally, IVR has relied primarily on visual displays and head-hands movement tracking to create mediated experiences. However, the utilization of VR head-mounted displays in combination with ECG and haptic devices presents new challenges, in practical, technical and in terms of how it agrees with existing literature \cite*{Klotzsche2023}.

Can mediated experiences of touch yield similar results to those observed in embodied cognition? By understanding which aspects of touch can be effectively mediated and how they interact with other senses in a mediated environment, we can enhance experimental setups to explore the body-mind relationship with greater ecological validity. Additionally, this understanding will provide valuable guidelines to the industry for the development of hardware devices in this domain.

\section*{2. Thesis Topic \& Goal}

The primary objective of this study is to investigate the feasibility of incorporating touch-cardiac-cycle modulation studies into Interactive Virtual Reality (IVR) setups. IVR, being a system that often involves visual, tactile, and proprioceptive senses, inherently engages multiple senses or is intentionally designed as a multisensory experience. To facilitate a comparative analysis of results, a relevant recent study by Martina Saltafossi on vision, touch, and hearing as multisensory pairs \cite{SALTAFOSSI2023108642} serves as a suitable reference. While there is limited research on two multisensory modalities and none to my knowledge using IVR, this study aims to bridge that gap. However, before delving into specific goals, it is necessary to define the concept of touch, as it encompasses various modes.

The extended classification of tactile sensation \cite{Healy2003HandbookOP} provides a useful framework for understanding touch, categorizing it into five different modes based on the presence or absence of voluntary movement: (1) tactile (cutaneous) perception, (2) passive kinesthetic perception, (3) passive haptic perception, (4) active kinesthetic perception, and (5) active haptic perception . For this thesis, touch is defined as passive haptic perception generated by a vibrating Data-Glove. Based on this definition, three main goals are derived:

\begin{itemize}
    \item[(i)] Assess the impact of passive haptic stimuli on the reported sense of immersion in individuals. This investigation aims to quantify the extent to which passive haptic stimuli influence overall reported scores in questionnaires, shedding light on the role of touch in creating a sense of presence. Saltafossi study refers to this as "body illusions induced by multisensory conflicts between exteroceptive sensory modalities, such as vision and touch."
    
    \item[(ii)] Evaluate the effect of passive haptic stimuli on performance in the motor-memory task. By examining how passive haptic stimuli influences the response time and accuracy in the motor-memory task, this study seeks to reveal the influence of touch on overall behavioral outcomes in the task. 
    
    \item[(iii)] If the preceding steps yield positive results, we will test if the unlocked-stimuli triggered at diastole or systole has any effect on the response times. This goal involves reproducing existing research that identified modulations in haptic perception synchronized with the cardiac cycle, contributing to the understanding and testing of VR head-mounted displays in combination with ECG and haptic devices.
\end{itemize}

Through an investigation of passive haptic touch's influence on immersion, behavioral outcomes, and interactions with the cardiac cycle, this research aims to enhance our understanding of IVR as a research tool and further validate findings on multisensory integration and perception.  

\section*{3. Methods}

There were a total of 23 participans who performed both of the full tasks. Nonetheless, 2 of them did not filled all questionaires therefore we had to leave them out and continue to performe the analysis with 21 participants. The call for participants was answer considerably more by women (17) than men (5). The age accorss the sample was consistenly around 25 years old, all but for one exception ($\mu=25.1 , \sigma=6.3$).

\section*{4. Materials}
The experimental design methods and materials used in this study are developed based on the NRO-228 Study-DB: 02188.07 - TSVR Akbal/Villringer. However, not all the material gathered in the original study will be utilized for this master's thesis. The heartbeat count task will be excluded to maintain focus on the specific goals of this thesis.

\begin{enumerate}
    \item[4.1] \textbf{Experimental Process:} Upon receiving information about the experiment, participants completed the following tasks:
    \begin{enumerate}
        \item[(i)] Edinburgh Handedness Questionnaire,
        \item[(ii)] PRE-Cybersickness Questionnaire
    \end{enumerate}

    Following the completion of the first set of Questionnaires the participant were brought to another room were after sitting, the IVR equipment is placed on them. This was a head-mounted display (HMD) and data gloves and the ECG device. Participants performed a short training round and started with a heartbeat count task and IVR memory-motor task.

    \begin{enumerate}
      \item[(iii)] Heart Beat Count Task (HCT) for one minute. (Not considered in this thesis.)
      \item[(iv)] IVR motor-memory Task: Within this environment, participants were positioned in front of a virtual table and given sufficient time to acclimate to the virtual surroundings. A sketch of a puzzle, which they needed to memorize, was displayed in front of the virtual room. During the memorization phase, participants kept their virtual hands open with palms facing up. Subsequently, the sketch disappeared from their visual field. A red ball was then introduced from the top, appearing in either one of the hands. Participants observed a template on the table, which was an identical representation of the initial sketch they had memorized. Their task was to place the ball in the correct location on the template as quickly as possible. Throughout 108 trials, three different conditions (relevant haptic stimuli, irrelevant haptic stimuli, and no haptic stimuli) were randomly presented an equal number of times (36 times each). The trials appeared rapidly, one after the other. The start of each trial was marked by placing the ball in a hand, and the end of the trial was marked by the moment the ball reached the designated location on the template. 
    \end{enumerate}

The final step of the experiment involved two questionnaires aimed at extracting information on several relevant fields for the goal of this thesis.
\begin{enumerate}
\item[(v)] Virtual Reality Subjective Evaluation Questionnaire
\item[(vi)] POST-Cybersickness Questionnaire
\end{enumerate}
     
    \item[4.2] \textbf{Immersive Virtual Reality Equipment:} The VR experience was presented and tracked using an HTC Vive head-mounted display (HMD), two lighthouses, a Leap Motion sensor, and Haptic Data Gloves. Movement data was collected from the data gloves, Leap Motion device, and the HMD. The data gloves utilized a magnetic tracking sensor for the fingertips, although this data was not considered for the movement analysis. Only the wrist movements, tracked by the Leap Motion device, were taken into account. All movements were recorded in a Euclidean coordinate system (X, Y, Z). The original calibrating point was set at (0, 0, 0), providing a total of nine streaming sources of data (e.g., Headset X, Headset Y, Headset Z, and so on). Rotational data was not included in the analysis.
    \item[4.3] \textbf{Electrocardiogram Equipment:} Heart rate data was collected using an Arduino Uno and a SparkFun Single Lead Heart Rate Monitor - AD8232. The collected data was integrated into the Unity log file at a frequency of 133 Hz.

\item[4.4] \textbf{Questionnaires Material:}
  \begin{enumerate}
    \item[(i)] Virtual Reality Subjective Evaluation Questionnaire: This self-made questionnaire consists of 26 items oriented towards capturing whether the experience felt real for the participant or not. It explores the level of engagement, hand movement, task difficulty, and other controlling factors. 
    \item[(ii)] PRE/POST-Cybersickness Questionnaire: The version applied in this study is a shorter adaptation of the simulator sickness questionnaire (SSQ) \cite{SSQ93}. The modified version provides straightforward computer or manual scoring, increased power to identify problem simulators, and improved diagnostic capability. It is administered before and after the IVR experiment.
\end{enumerate}
  \end{enumerate}


\pagebreak
\paragraph{\textbf{References}}
\printbibliography[heading=none]
\end{document}